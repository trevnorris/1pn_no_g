\documentclass[11pt]{article}
\usepackage[margin=1in]{geometry}
\usepackage{amsmath,amssymb,bm,mathtools}
\usepackage{physics}
\usepackage{amsthm}
\usepackage{hyperref}
\hypersetup{colorlinks=true,linkcolor=blue,citecolor=blue,urlcolor=blue}
\usepackage{mdframed}
\usepackage{enumitem}
\usepackage{tikz}
\usepackage{booktabs}
\usepackage{siunitx}
\setlist{nosep}
\numberwithin{equation}{section}

\newenvironment{resultbox}{\begin{mdframed}[linewidth=0.8pt,roundcorner=6pt]}{\end{mdframed}}

\theoremstyle{plain}
\newtheorem{proposition}{Proposition}[section]
\newtheorem{lemma}[proposition]{Lemma}

\theoremstyle{remark}
\newtheorem*{remark}{Remark}

\theoremstyle{definition}
\newtheorem{definition}[proposition]{Definition}

\title{Superfluid Slab Gravity Without \texorpdfstring{$G$}{G}: \\
Exact Newtonian Orbits and Conservative 1PN--Like Effects from a Single Superfluid Equation \\[4pt]
\large (Corrected and clarified edition)}
\author{VT-7 Collaboration}
\date{\today}

\begin{document}
\maketitle

\section*{Section 0: Reader's guide and main claims}
\label{sec:claims}
\paragraph{Scope and stance.}
We make no ontological claim about the nature of gravity; we simply show that Newtonian gravity and 1PN--like orbital structures follow from superfluid hydrodynamics on a slab with appropriate boundary conditions.

\paragraph{What we assume (one equation, plus mouths).}
A single superfluid lives on a 3D slab with complex order parameter $\psi=\sqrt{\rho}\,e^{i\theta}$. Bodies are modeled as tiny \emph{intakes} (\emph{mouths}) that remove fluid locally at rates $q_a(t)$. In hydrodynamic variables $(\rho,\mathbf v)$ this gives continuity with sink terms and the inviscid superfluid Euler/Bernoulli equation (including quantum pressure). There is \emph{no} gravitational field and \emph{no} $G$ in the theory; the orbital constant that appears is $K\equiv \rho_0/(4\pi\beta^2)$.

\paragraph{How gravity appears (derived, not assumed).}
Outside the tiny mouths, take the \emph{ideal} exterior limit: uniform ambient density, irrotational flow, negligible dispersion. The velocity field is potential and satisfies a Poisson equation with point sinks. Computing the force as \emph{external} momentum flux through a small control surface around each mouth yields $\mathbf F_a=\rho_0\,Q_a\,\mathbf v_{\rm ext}(\mathbf r_a)$ (control-surface lemma). Identifying inertial mass with throughput, $M_a=\beta Q_a$, makes accelerations independent of the test body (automatic equivalence principle) and reproduces \emph{exact} Newtonian $N$-body motion with the single constant $K=\rho_0/(4\pi\beta^2)$.

\paragraph{What ``1PN-like'' means here (same equation, small terms kept).}
Keeping small effects already present in the same equation---finite sound speed $c_s$ (retarded response), convective/Bernoulli nonlinearity (field self-energy), dispersion via the healing length $\xi$, and finite mouth size $a_0$---produces a compact, \emph{conservative} correction to the two-body dynamics with the same structures as the familiar 1PN orbital sector (perihelion advance, velocity-dependent central terms). No new fields are added; all coefficients are functions of $(c_s,\xi,a_0)$ and the equation of state near $\rho_0$.

\paragraph{One non-gravitational datum fixes the absolute scale.}
Orbits determine only $K=\rho_0/(4\pi\beta^2)$, not $\rho_0$ and $\beta$ separately. A single non-gravitational datum (e.g., a bound/measurement of $c_s$, $\xi$, or a low-frequency impedance) fixes the universal flow-per-mass $Q/M=1/\beta$, turning masses into absolute intakes for simulation.

\paragraph{Universality and where it can fail (capacity).}
The mapping $M=\beta Q$ is universal when mouths are far below sonic capacity. Define the surface Mach number $\mathrm{Ma}=v(R)/c_s=Q/(4\pi R^2 c_s)$. If $\mathrm{Ma}\ll1$, Newton is recovered and any rarefied halo is $\propto r^{-4}$ and tiny. Near $\mathrm{Ma}\sim1$, the intake \emph{chokes} at $Q_{\rm crit}\sim4\pi R^2\rho_0 c_s$, giving a controlled route to non-Newtonian ``saturation'' in extreme objects.

\paragraph{What we \emph{do not} attempt here.}
We do not treat light, lensing, or radiation reaction. Electromagnetism and photon propagation live in the bulk of the broader program and are unnecessary to obtain conservative orbital phenomenology on the slab.

\paragraph{Key symbols.}
$\rho_0$: ambient density; $c_s$: sound speed; $\xi$: healing length; $a_0$: mouth size; $Q$: volumetric intake; $\beta$: mass--intake factor ($M=\beta Q$); $K$: Newton-form constant $\rho_0/(4\pi\beta^2)$.

\section{Introduction and scope}
\label{sec:intro}
We derive Newtonian $N$-body gravity, and its conservative 1PN-like orbital corrections, from a \emph{single} superfluid equation on a 3D slab with localized \emph{intakes} (``mouths''). No gravitational field or constant is assumed: bodies draw slab fluid at rates $q_a$, and the exterior flow exerts forces obtained by momentum flux.
In the ideal exterior regime (uniform density, irrotational flow, negligible dispersion), the velocity potential solves a Poisson equation with point sinks; evaluating the external momentum flux across a small sphere surrounding each mouth yields a thrust $\mathbf F=\rho_0 Q\,\mathbf v_{\rm ext}$. Identifying inertial mass as throughput, $M=\beta Q$, produces the exact Newtonian law with a single combination $K=\rho_0/(4\pi\beta^2)$. Keeping the small terms of the \emph{same} equation (finite $c_s$, dispersion $\xi$, finite mouth size $a_0$) generates a compact, conservative two-body Lagrangian with the same structures that control 1PN orbital phenomenology.

\section{Unreduced superfluid slab: equation, mouths, and observables}
\label{sec:model}

\subsection{Governing equation (no $G$)}
The slab is modeled by a complex order parameter $\psi(\mathbf x,t)=\sqrt{\rho(\mathbf x,t)}\,e^{i\theta(\mathbf x,t)}$ obeying a Gross--Pitaevskii--type evolution with localized \emph{intakes}:
\begin{equation}
i\,\hbar_*\,\partial_t\psi
=\Big[-\frac{\hbar_*^2}{2m_*}\nabla^2+\mu'(\rho)\Big]\psi
\;+\; i\sum_{a=1}^N q_a(t)\,W_a\!\big(\mathbf x-\mathbf r_a(t)\big)\,\psi .
\label{eq:GP}
\end{equation}
Here $\rho=|\psi|^2$; $\mathbf v=\frac{\hbar_*}{m_*}\nabla\theta$ (irrotational away from vortex cores); $\mu'(\rho)$ derives from a barotropic EOS with $c_s^2=\rho_0\,\mu''(\rho_0)$;  $\xi=\hbar_*/(\sqrt2\,m_*c_s)$ is the healing length; each mouth $a$ removes fluid at rate $q_a(t)$ through a normalized shape $W_a(\mathbf x)$ supported in a ball of radius $a_0$.

\subsection{Hydrodynamic form (Madelung split)}
Write $\psi=\sqrt\rho\,e^{i\theta}$ and separate real/imaginary parts:
\begin{align}
\partial_t\rho+\nabla\!\cdot(\rho\,\mathbf v)&= -\sum_a q_a(t)\,W_a(\mathbf x-\mathbf r_a),
\label{eq:cont}\\
\partial_t\mathbf v+(\mathbf v\!\cdot\!\nabla)\mathbf v&= -\nabla\!\Big[h(\rho)+Q(\rho)\Big],
\label{eq:euler}
\end{align}
with enthalpy $h'(\rho)=\mu'(\rho)/m_*$ and quantum-pressure potential
$Q(\rho) = -\frac{\hbar_*^2}{2m_*^2}\,\frac{\nabla^2\sqrt{\rho}}{\sqrt{\rho}}$.
In the exterior (away from tiny mouths) the flow is irrotational, so $(\mathbf v\!\cdot\!\nabla)\mathbf v=\nabla(\tfrac12|\mathbf v|^2)$.

\subsection{Flux conventions, external flow, and force}
\label{sec:flux}
We define the volumetric intake with the \emph{outward} normal $\mathbf n$ on a small sphere $\partial\mathcal B_a$ of radius $\varepsilon$ around mouth $a$ as
\begin{equation}
Q_a(t)\equiv -\oint_{\partial\mathcal B_a}\mathbf v\!\cdot \mathbf n\,\,dA,
\qquad Q_a=\frac{q_a}{\rho_0}\,,
\label{eq:Qdef}
\end{equation}
so that $Q_a>0$ corresponds to an inward (sink) flow. The net force on body $a$ is the \emph{external} momentum flux through $\partial\mathcal B_a$ (self-field subtracted):
\begin{equation}
\mathbf F_a \;=\; \oint_{\partial\mathcal B_a}
\Big[\rho\,\mathbf v(\mathbf v\!\cdot\!\mathbf n) - \big(P(\rho)+\tfrac12\rho|\mathbf v|^2\big)\,\mathbf n\Big]\,dA \;+\; \mathbf F_a^{(Q)}.
\label{eq:force}
\end{equation}
In the exterior ideal limit ($\rho\to\rho_0$), gradients of $\sqrt\rho$ vanish and with them the quantum-pressure surface contribution: $\mathbf F^{(Q)}_a=0$. In Sec.~\ref{sec:ideal} we evaluate \eqref{eq:force} explicitly and obtain the control-surface lemma.

\paragraph{Scope note (bulk drain).}
The removed fluid drains into an orthogonal 4D bulk (not modeled here); we focus solely on the slab dynamics where bodies reside.

\section{Ideal limit and exact Newton mapping (no $G$)}
\label{sec:ideal}

\subsection{Potential-flow equation (sinks have positive potential)}
Under the exterior ideal assumptions (uniform $\rho\to\rho_0$, irrotational flow, negligible dispersion), continuity gives
\begin{equation}
\nabla\!\cdot \mathbf v = -\sum_a Q_a\,\delta^3(\mathbf x-\mathbf r_a).
\label{eq:divv}
\end{equation}
With $\mathbf v=\nabla\phi$ we obtain Poisson's equation
\begin{equation}
\nabla^2 \phi(\mathbf x,t) = -\sum_{a=1}^N Q_a(t)\,\delta^3\!\big(\mathbf x-\mathbf r_a(t)\big),
\qquad
\phi(\mathbf x,t) = +\sum_a \frac{Q_a(t)}{4\pi\,|\mathbf x-\mathbf r_a(t)|},
\label{eq:poisson}
\end{equation}
since $\nabla^2(1/r)=-4\pi\delta^3$. Sinks ($Q_a>0$) thus have \emph{positive} potential and inward velocity $\mathbf v=\nabla\phi$.

\subsection{Near-mouth expansion and momentum-flux force}
Near mouth $a$ the velocity splits into an isotropic self-field plus the external field:
\begin{equation}
\mathbf v(\mathbf x)= -\frac{Q_a}{4\pi r^2}\,\hat{\mathbf r}+\mathbf v_{\rm ext}(\mathbf r_a)+O(r),\qquad
r=|\mathbf x-\mathbf r_a|.
\label{eq:near}
\end{equation}
Inserting \eqref{eq:near} into \eqref{eq:force} and subtracting the self term leaves
\begin{equation}
\boxed{\ \mathbf F_a = \rho_0 Q_a\,\mathbf v_{\rm ext}(\mathbf r_a)\ }\qquad\text{(control-surface lemma; explicit derivation in App.~\ref{app:flux}).}
\label{eq:controlsurface}
\end{equation}
Using \eqref{eq:poisson},
\begin{equation}
\boxed{\ \mathbf F_a \;=\; \sum_{b\neq a} \frac{\rho_0}{4\pi}\,\frac{Q_a Q_b}{r_{ab}^2}\,\hat{\mathbf r}_{ab}\ },
\qquad \hat{\mathbf r}_{ab}\equiv\frac{\mathbf r_b-\mathbf r_a}{r_{ab}}.
\label{eq:pair_force}
\end{equation}

\subsection{Mass--intake map and Newton form}
Define the mass--intake relation
\begin{equation}
M_a=\beta\,Q_a, \qquad [\beta]=\mathrm{kg\,s/m^3},
\label{eq:massmap}
\end{equation}
then
\begin{equation}
\ddot{\mathbf r}_a \;=\; \sum_{b\neq a} K\,M_b\,\frac{\mathbf r_{b}-\mathbf r_{a}}{r_{ab}^3},
\qquad \boxed{\,K\equiv \frac{\rho_0}{4\pi\beta^2}\,}.
\label{eq:newton_form}
\end{equation}
No $G$ appears; orbital data determine only $K$.

\subsection{Energy and Kepler checks}
The field energy equals the pair interaction $U_{\rm fluid}=-\frac{\rho_0}{4\pi}\sum_{a<b}Q_aQ_b/r_{ab}=-K\sum_{a<b}M_aM_b/r_{ab}$. With $T=\tfrac12\sum_a M_a v_a^2$, $E=T+U_{\rm fluid}$ is conserved. For two bodies, $\ddot{\mathbf r}=-K(M_1+M_2)\mathbf r/r^3$ and $n^2 a^3=K(M_1+M_2)$, i.e. Kepler’s third law with $K$.

\section{Beyond-ideal, same equation: small-parameter expansion}
\label{sec:expansion}
Define small parameters $\varepsilon=v/c_s\ll1$, $\chi=\xi/r\ll1$, $\delta=a_0/r\ll1$.
Keeping the corresponding terms in the \emph{same} equations \eqref{eq:cont}--\eqref{eq:force} yields a conservative two-body Lagrangian
\begin{equation}
L_{\rm eff} = \frac12\,\mu\,v^2 + \frac{K\,M\mu}{r}
+ \frac{K\,M\mu}{2\,c_s^2\,r}\Big[ A_1\,v^2 + A_2\,(\hat{\mathbf n}\!\cdot\!\mathbf v)^2 \Big]
- K\,M\mu\,\frac{B_\xi\,\xi^2+B_a\,a_0^2}{r^3}
+ O(\varepsilon^3,\chi^4,\delta^4),
\label{eq:Leff}
\end{equation}
with $M=M_1+M_2$, $\mu=M_1M_2/M$, $\hat{\mathbf n}=\mathbf r/r$.
Here: (i) $A_{1,2}=O(1)$ arise from finite sound speed (retardation) and convective nonlinearity; (ii) $B_\xi=O(1)$ arises from dispersion (quantum pressure); (iii) $B_a=O(1)$ parameterizes finite-mouth geometry (central term vanishes for spherical mouths in free space; see App.~\ref{app:mouth}).
For smooth EOS and compact $W(\mathbf x)$ one has $|A_i|,|B_{\xi,a}|\lesssim \mathrm{few}$.

For bound motion (semi-major axis $a$, eccentricity $e$), the perihelion advance is
\begin{equation}
\Delta\varpi =\frac{2\pi}{1-e^2}\left[\frac{K M}{a\,c_s^2}\,\mathcal A +\frac{\mathcal B_\xi\,\xi^2+\mathcal B_a\,a_0^2}{a^2}\right]
+ O(\varepsilon^3,\chi^4,\delta^4),
\label{eq:precession}
\end{equation}
where the velocity-dependent pieces $\propto v^2/c_s^2$ yield the $KM/(a c_s^2)$ factor with $\mathcal A=O(1)$, and the static $r^{-3}$ pieces from $\xi$ and $a_0$ yield $(\xi^2+a_0^2)/a^2$ with $\mathcal B_{(\cdot)}=O(1)$.

\section{Fixing the universal flow--per--mass with one non-gravitational datum}
\label{sec:calibration}
Orbits fix only $K=\rho_0/(4\pi\beta^2)$. The absolute throughput follows from
\begin{equation}
\boxed{\ \frac{Q}{M}=\frac{1}{\beta}=\sqrt{\frac{4\pi K}{\rho_0}}\ },
\qquad \text{(requires one non-gravitational datum to fix $\rho_0$ or $\beta$).}
\end{equation}
\paragraph{Calibration as a procedure.}
\begin{enumerate}[label=\arabic*)]
\item Fit orbits $\Rightarrow$ determine $K$.
\item Measure a non-gravitational datum $X$ (e.g., $c_s$, $\xi$, low-frequency impedance).
\item Use the EOS to infer $\rho_0$ from $X$.
\item Compute $\beta=\sqrt{\rho_0/(4\pi K)}$ and thus $Q/M=1/\beta$.
\item All absolute throughputs $Q$ and flow fields are now fixed (capacity checks included).
\end{enumerate}

\section{Falsifiability and bounds}
\label{sec:falsify}
Conservative diagnostics: perihelion precession \eqref{eq:precession} and Lagrange-point shifts scale with the same combinations $K M/(a c_s^2)$ and $(\xi^2+a_0^2)/a^2$. Dissipative diagnostics (used only for bounds): any residual shear drag ($\mathbf F_d=-\gamma\mathbf v$) gives $\gamma<\mu|\dot a|/(2a)$ from an observed limit on $|\dot a|$; acoustic leakage power scales as $P_{\rm ac}\sim \rho_0 Q^2\Omega^4 a^2/c_s^3$ and yields a lower bound on $c_s$. Finite slab thickness $H$ introduces tiny near-field corrections $\propto r^2/H^3$ and, when coupled to finite $a_0$, a central renormalization $\propto (a_0/H)^2$ (App.~\ref{app:slab}). Vorticity would induce torque noise on orbits; current ephemerides tightly bound such noise.

\section{Numerical scheme}
\label{sec:numerics}
Minimal loop:
\begin{enumerate}[label=\arabic*)]
\item Store $\{\mathbf r_a,\mathbf p_a,M_a,Q_a,R_a,\texttt{shape}_a\}$ with $M_a=\beta Q_a$.
\item Compute fields either by closed-form potential flow (ideal mode) or by discretizing \eqref{eq:cont}--\eqref{eq:euler}.
\item Evaluate forces by momentum flux with analytic self-field subtraction (App.~\ref{app:flux}).
\item Integrate with symplectic KDK or Wisdom--Holman.
\item Optional knobs: finite-$c_s$ (retarded evaluation to $O(\varepsilon^2)$), dispersion ($\xi$), finite mouth ($a_0$), slab thickness ($H$), capacity model $Q^{\rm eff}=Q^{\rm nom}/\sqrt{1+(Q^{\rm nom}/4\pi R^2c_s)^2}$.
\item Diagnostics: invariants ($E,\mathbf L$), precession, Lagrange points, $\epsilon$-plateau for control surface, A/B force-path equality.
\end{enumerate}

\paragraph{Pseudocode (ideal mode).}
\begin{verbatim}
for each timestep:
    1. phi(x,t) = + sum_a Q_a / (4*pi*|x - r_a|)          # sinks => positive potential
    2. v_ext(r_a) = sum_{b != a} grad phi_b(r_a)
    3. F_a = rho0 * Q_a * v_ext(r_a)
    4. Update (r_a, p_a) via symplectic step
    5. Monitor E, L, precession, control-surface epsilon-plateau
\end{verbatim}

\section{Structural correspondence to metric 1PN (dictionary only)}
\label{sec:dictionary}
Our conservative Lagrangian \eqref{eq:Leff} matches the \emph{structure} of the EIH 1PN two-body Lagrangian: identify $K\leftrightarrow G$ and, optionally, $c_s\leftrightarrow c$ to recover GR coefficient values (e.g., $\mathcal A=3$ in the test-body limit). Medium fingerprints without a GR analog are the $r^{-3}$ pieces $\propto \xi^2$ and $a_0^2$, which should be bounded small if exact GR matching is desired.

\section{Conclusions and outlook}
\label{sec:conclusion}
A single superfluid equation on a 3D slab with localized intakes yields exact Newtonian $N$-body motion without $G$. Keeping small, already-present terms produces conservative, 1PN-like orbital effects with parameters intrinsic to the slab ($c_s,\xi,a_0$). Dissipative channels are naturally tiny and serve to bound parameters. A minimal simulator and a clear test plan render the framework falsifiable. An optional dictionary to GR is a matter of parameter choices, not extra structure.

\appendix

\section{Control–surface derivation of $\mathbf F=\rho_0 Q\,\mathbf v_{\rm ext}$}
\label{app:flux}
We present the explicit evaluation of the surface integral in \eqref{eq:force} at linear order in the external field.
Consider a small sphere $\partial\mathcal B_\varepsilon$ of radius $\varepsilon$ centered on a mouth at $\mathbf r_a$.
Let $\hat{\mathbf n}$ be the outward normal and write the near-field decomposition
$\mathbf v=\mathbf v_{\rm self}+\mathbf v_{\rm ext}+O(\varepsilon)$ with
$\mathbf v_{\rm self}=-(Q_a/4\pi\varepsilon^2)\hat{\mathbf n}$ as in \eqref{eq:near}.
The inviscid momentum flux through $\partial\mathcal B_\varepsilon$ is
\begin{equation}
\mathbf F^{(\mathrm{HD})}
=\oint \Big[\rho_0\,\mathbf v(\mathbf v\!\cdot\!\hat{\mathbf n})
-\big(P+\tfrac12\rho_0 v^2\big)\hat{\mathbf n}\Big]\,dA.
\end{equation}
Keeping terms linear in $\mathbf v_{\rm ext}$ and using the angular averages
$\int \hat n_i\,d\Omega=0$ and $\int \hat n_i \hat n_j\,d\Omega=\tfrac{4\pi}{3}\delta_{ij}$, one finds
\begin{align}
\oint \mathbf v_{\rm self}(\mathbf v_{\rm ext}\!\cdot\!\hat{\mathbf n})\,dA &= -\frac{Q_a}{3}\,\mathbf v_{\rm ext},\\
\oint \mathbf v_{\rm ext}(\mathbf v_{\rm self}\!\cdot\!\hat{\mathbf n})\,dA &= -Q_a\,\mathbf v_{\rm ext},\\
\oint \big(\mathbf v_{\rm self}\!\cdot\!\mathbf v_{\rm ext}\big)\hat{\mathbf n}\,dA &= -\frac{Q_a}{3}\,\mathbf v_{\rm ext}.
\end{align}
The pressure/kinetic combination contributes $+\tfrac{Q_a}{3}\mathbf v_{\rm ext}$, cancelling the corresponding isotropic part so that
$\mathbf F^{(\mathrm{HD})}=-\rho_0 Q_a\,\mathbf v_{\rm ext}$.
Adopting the intake convention \eqref{eq:Qdef}, the total force on the \emph{body} is
\begin{equation}
\boxed{\ \mathbf F=\rho_0\,Q_a\,\mathbf v_{\rm ext}(\mathbf r_a)\ }.
\end{equation}
Equivalently, one may attribute the final sign to the quantum/Madelung stress surface term or to a finite-part subtraction of the isotropic self-field; all routes yield the same control-surface lemma.

\section{Two–body effective Lagrangian through $O(\varepsilon^2,\chi^2,\delta^2)$}
\label{app:lagrangian}
We sketch the steps; explicit algebra is standard and yields \eqref{eq:Leff}.

\subsection*{Finite $c_s$ (retardation)}
Use the retarded Green function for the linearized potential and expand the retarded time $t-R/c_s$ to $O(v^2/c_s^2)$. The resulting Lagrangian correction is
\begin{equation*}
\Delta L_{c_s}= \frac{K M_1M_2}{2 c_s^2 r}\big[\alpha_1(v_1^2+v_2^2)+\alpha_2\,\mathbf v_1\!\cdot\!\mathbf v_2+\alpha_3(\hat{\mathbf n}\!\cdot\!\mathbf v_1)(\hat{\mathbf n}\!\cdot\!\mathbf v_2)\big],
\end{equation*}
with $\alpha_i=O(1)$.

\subsection*{Convective/Bernoulli nonlinearity}
Eliminate $\rho'$ with the EOS to get a quadratic energy density $\propto |\nabla\phi|^2/c_s^2$, yielding $\Delta U_{\rm conv}=-(K M_1M_2/r)\,\alpha_4\,v_{\rm rel}^2/c_s^2$, with $\alpha_4=O(1)$.

\subsection*{Dispersion}
Include the $k^4$ piece from $Q(\rho)$ to get a static propagator $G(r)=(4\pi r)^{-1}[1+\xi^2/r^2+\cdots]$, i.e.\ $\Delta U_{\xi}=-K M_1M_2\,B_\xi\,\xi^2/r^3$, $B_\xi=O(1)$.

\subsection*{Finite mouth}
Multipole-expand a compact mouth shape $W(\mathbf x)$ with radius $a_0$. Anisotropy produces a quadrupolar $r^{-3}$ piece $\propto \mathcal Q_{ij}\hat n_i\hat n_j$ which averages to zero unless aligned; a central proxy is $-K M_1M_2\,B_a a_0^2/r^3$ for persistent alignment or boundary-coupled symmetry breaking, with $B_a=O(1)$.

\section{Finite–mouth multipoles and $(a_0/r)^2$ corrections}
\label{app:mouth}
Let $W(\mathbf x)$ be a compact mouth of size $a_0$ and define $\Sigma_{ij}=\int x_i x_j W\,d^3x$, $I_2=\mathrm{Tr}\,\Sigma$, and the traceless quadrupole $\mathcal Q_{ij}=3\Sigma_{ij}-I_2\delta_{ij}$. For $R\gg a_0$,
\begin{equation}
\phi_a(\mathbf R)=-\frac{Q_a}{4\pi}\left[\frac{1}{R}+\frac{1}{8R^3}\,\mathcal Q_{ij}\hat R_i\hat R_j+\cdots\right].
\end{equation}
A \emph{spherical} mouth has $\mathcal Q_{ij}=0$ and thus an exactly monopolar exterior field (no $(a_0/r)^2$ correction in free space). Two-body interaction:
\begin{equation}
U=-K\frac{M_1M_2}{r}-\frac{K M_2}{8 r^3}\mathcal Q^{(1)}_{ij}\hat n_i\hat n_j-\frac{K M_1}{8 r^3}\mathcal Q^{(2)}_{ij}\hat n_i\hat n_j+\cdots.
\end{equation}
Angle averages kill the quadrupole unless there is persistent alignment (e.g., tidal locking).

\section{Slab Green’s functions and finite thickness}
\label{app:slab}
For Neumann walls at $z=0,H$, the Green function is
\begin{equation}
G_N(\bm\rho,z,z')=\frac{1}{4\pi}\sum_{m=-\infty}^{\infty} \left[
\frac{1}{\sqrt{\rho^2+(z-z'+2mH)^2}}
+\frac{1}{\sqrt{\rho^2+(z+z'-2mH)^2}} \right].
\end{equation}
Near-field ($r\ll H$): $G_N=\frac{1}{4\pi r}-\frac{\zeta(3)}{4\pi}\frac{r^2}{H^3}+O(r^4/H^5)$ (midplane). Thus $U=-K M_1M_2/r + K M_1M_2\,\frac{\zeta(3)}{4\pi}\,r^2/H^3+\cdots$. A central $1/r$ renormalization arises only when coupled with finite $a_0$ as $(a_0/H)^2$ (images shift the intake/pressure relation at the throat).

\section{Units, dimensions, and post–hoc comparison to Newtonian parameters}
\label{app:units}
Dimensions (SI): $[\rho_0]=\si{kg.m^{-3}}$, $[c_s]=\si{m.s^{-1}}$, $[\xi]=\si{m}$, $[a_0]=\si{m}$, $[Q]=\si{m^3.s^{-1}}$, $[\beta]=\si{kg.s.m^{-3}}$, $[K]=\si{m^3.kg^{-1}.s^{-2}}$.
Dictionary to Newton (optional): identify $K\leftrightarrow G$ and, if desired, $c_s\leftrightarrow c$.
Throughput per mass: $\dfrac{Q}{M}=\dfrac{1}{\beta}=\sqrt{\dfrac{4\pi K}{\rho_0}}$.
Capacity threshold surface gravity (from $v(R)=c_s$): $g_\star=2\sqrt{\pi}\,c_s\,\sqrt{K\rho_0}$.

\paragraph{Dimensional check for $L_{\rm eff}$ terms in \eqref{eq:Leff}.}
\begin{center}
\begin{tabular}{@{}ll@{}}
\toprule
Term & Units \\
\midrule
$\tfrac12 \mu v^2$ & $\si{kg.m^2.s^{-2}}$ \\
$K M\mu / r$ & $[\si{m^3.kg^{-1}.s^{-2}}][\si{kg}][\si{kg}]/[\si{m}]=\si{kg.m^2.s^{-2}}$ \\
$K M\mu v^2/(c_s^2 r)$ & same as above $\to \si{kg.m^2.s^{-2}}$ \\
$K M\mu \xi^2/r^3$ & $[\si{m^3.kg^{-1}.s^{-2}}][\si{kg}][\si{kg}][\si{m^2}]/[\si{m^3}]=\si{kg.m^2.s^{-2}}$ \\
\bottomrule
\end{tabular}
\end{center}

\end{document}
